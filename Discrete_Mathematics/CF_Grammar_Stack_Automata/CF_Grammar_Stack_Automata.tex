% !TEX TS-program = pdflatex
% !TEX encoding = UTF-8 Unicode

% This is a simple template for a LaTeX document using the "article" class.
% See "book", "report", "letter" for other types of document.

\documentclass[11pt]{article} % use larger type; default would be 10pt

\usepackage[utf8]{inputenc} % set input encoding (not needed with XeLaTeX)

%%% Examples of Article customizations
% These packages are optional, depending whether you want the features they provide.
% See the LaTeX Companion or other references for full information.

%%% PAGE DIMENSIONS
\usepackage{geometry} % to change the page dimensions
\geometry{a4paper} % or letterpaper (US) or a5paper or....
% \geometry{margin=2in} % for example, change the margins to 2 inches all round
% \geometry{landscape} % set up the page for landscape
%   read geometry.pdf for detailed page layout information

\usepackage{graphicx} % support the \includegraphics command and options

% \usepackage[parfill]{parskip} % Activate to begin paragraphs with an empty line rather than an indent

%%% PACKAGES
\usepackage{booktabs} % for much better looking tables
\usepackage{array} % for better arrays (eg matrices) in maths
\usepackage{paralist} % very flexible & customisable lists (eg. enumerate/itemize, etc.)
\usepackage{verbatim} % adds environment for commenting out blocks of text & for better verbatim
\usepackage{subfig} % make it possible to include more than one captioned figure/table in a single float
% These packages are all incorporated in the memoir class to one degree or another...

%%% HEADERS & FOOTERS
\usepackage{fancyhdr} % This should be set AFTER setting up the page geometry
\pagestyle{fancy} % options: empty , plain , fancy
\renewcommand{\headrulewidth}{0pt} % customise the layout...
\lhead{}\chead{}\rhead{}
\lfoot{}\cfoot{\thepage}\rfoot{}

%%% SECTION TITLE APPEARANCE
\usepackage{sectsty}


\allsectionsfont{\sffamily\mdseries\upshape} % (See the fntguide.pdf for font help)
% (This matches ConTeXt defaults)

%%% ToC (table of contents) APPEARANCE
\usepackage[nottoc,notlof,notlot]{tocbibind} % Put the bibliography in the ToC
\usepackage[titles,subfigure]{tocloft} % Alter the style of the Table of Contents
\renewcommand{\cftsecfont}{\rmfamily\mdseries\upshape}
\renewcommand{\cftsecpagefont}{\rmfamily\mdseries\upshape} % No bold!

%%% END Article customizations


\usepackage[bulgarian]{babel}
\usepackage{physics}
\usepackage{amsmath}
\usepackage{centernot}
\usepackage{url}

%%% The "real" document content comes below...

\title{Контекстно свободни граматики. Стекови автомати}
\author{Play4u}
%\date{} % Activate to display a given date or no date (if empty),
         % otherwise the current date is printed 

\begin{document}
\maketitle

\newcommand{\lrangle}[1]{\left\langle #1 \right\rangle}

\newcommand{\belongsTo}{\in}
\newcommand{\notBelongsTo}{\centernot\in}

\newcommand{\kda}{A = <Q, X, q_{0}, \delta, F>}
\newcommand{\cfg}{\Gamma = <\mathcal{N}, \mathcal{T}, \mathcal{S}, \mathcal{P}>}
\newcommand{\cfgVers}{G = \langle V, \Sigma, R, S \rangle}
\newcommand{\nsa}{A = <Q, X, Z, q_{0}, z_{0}, \delta, F>}

\newcommand{\italicBold}[1]{\textbf{\emph{#1}}}
\newcommand{\definition}{\italicBold{Дефиниция: }}
\newcommand{\theorem}{\italicBold{Теорема: }}
\newcommand{\lemma}{\italicBold{Лема: }}
\newcommand{\proof}{\italicBold{Доказателство: }}

\newcommand{\curlies}[1]{\{#1\}}

\newcommand{\enumNum}{\renewcommand{\theenumi}{\arabic{enumi}}}
\newcommand{\enumlet}{\renewcommand{\theenumi}{\alph{enumi}}}

\section{Контекстно свободни граматики}
Всеки контекстно-свободен език се поражда от контекстно-свободна граматика $\cfg$, правилата на която са от вида $А \to \alpha, A \in N, \alpha \in (N \cup T)^{+}$.
Ако в граматиката няма аксиома в дясба част на правило, допустимо е и правилото $S \to \epsilon$, позволяващо ни да извеждаме само празната дума. \par

На всеки извод в контекстно-свободната граматика ще  съпоставим дървовидна структура със следната 

\definition{Нека $\cfg$ е контекстно-свободна граматика. Дървото $D(V, E)$ с корен $r \in V$, с наредба на синовете и функция $f: V \to (N \cup T)$< съпоставяща на всеки връх буква от азбуките на граматиката, наричаме \emph{дърво на извод} в $\Gamma$, ако 

\enumlet
\begin{enumerate}
	\item За всеки нелист $v$ на $D, f(v) \in N$, като $f(r) = S$;\\
	\item За всеки лист $l$ на $D, f(l) \in T$;\\
	\item За всеки нелист $v$ с наредена последвоателност от синове $v_{i_1}, v_{i_2}, ..., v_{i_k}$, е в сила $f(v_{i_1})f(v_{i_2})...f(v_{i_k})\in P$. 
\end{enumerate}

\italicBold{Бележка от автора: В конспекта за държавния изпит се споменава въпроса "Дърво на синтаксиса", докато в учебника по ДСТР присъства единствено понятието "Дърво на изводите" (което ще се използва и в този документ). Автора предполага, че двете са едно и също чрез следните доводи:} 

\enumNum
\begin{enumerate}
	\item Предполага, че превода на "извод" на англ. е "derivation".\\
	\item Според \path{https://en.wikipedia.org/wiki/Parse_tree} Syntax tree $\equiv$ Derivation tree \\
\end{enumerate} \par

Дума на \emph{дърво на извод} дефинираме индуктивно, следвайки една от дефинициите на понятието дърво.\\

\definition Нека $D(V, E)$ е дърво на извод в граматиката $\cfg$

\enumlet
\begin{enumerate}
	\item Ако $D$ е дърво от вида\\
		\centerline{$S$-------$\epsilon$}
		то дума на $D$ е $\epsilon$;\\
	\item Нека $D'(\curlies{l}, \curlies{})$ е поддърво на дървото на извод $D$, състоящо се само от лист $l$. Дума на $D'$ е $f(l)$;\\
	\item Нека $D'$ е поддърво на $D$ с корен $r', D_{1}, D_{2}, ..., D_{k}$ са всички поддървета на $D'$ с корени синовете на $r'$, подредену в съотвествие с наредбата на тези синове и $\alpha_{i}$ е думата на поддървото $D_{i}, i = 1, 2, ..., k$. Тогава думата на $D'$ е $\alpha_{1}, \alpha_{2}, ...,\alpha_{k}$. 
\end{enumerate} \par

\theorem Нека $\cfg$ е КСГ. Съществува дърво на извод $D$ в $\Gamma_{A}$ с дума $\omega$ ще покажем, че $A^{\Gamma} \models^{A} \omega$ \\
\centerline \emph{Insert proof here}

\definition Ако в КСГ $\Gamma$ съществува повече от едно дърво на извод за някоя дума $\omega$, тогава наричаме $\Gamma$ \emph{еднозначна} \\ 

\definition КСЕ(език) $L$, закойто всяка КСГ, която го поражда е нееднозначна, наричаме \emph{нееднозначен език}.

\section{Нормална форма на Чомски}

\italicBold{Не успях да намеря информация в учебника по ДСТР. Всичко в този раздел  е взето от \path{https://store.fmi.uni-sofia.bg/fmi/logic/stefanv/files/eai/eai-notes.pdf} стр. 57} \\ \par

\definition Една КСГ е в нормална форма на Чомски ако всяко правило е от вида \\
\centerline{$A \to BC$ и $A \to \alpha$}
Като $B$ и $C$ не могат да бъдат променливата за начало $S$\\ 
\centerline{\emph{Insert proof here}}



\section{Стекови автомати}
\subsection{Недетерминирани стекови автомати}
Както вече доказахме КДА не могат да разпознават КСЕ. Затова трябва да дефиниране по сложни машини, разпознаващи тези езици. Важното за тези езици е понятието \emph{стек}. Стекът е запомнящо устрпйство, в което могат да се записват буквите на някаква азбука $Z$, и да се вземат обратно от стека, когато е необходимо. Това означава, че всеки новозапомнен елемент се поставя над елемента, обозначен като връх на стека и сам става връх. Когато трябва да се вземе елемент от стека, това непременно е върхът и връх става елементът, който е влязъл непосредствено преди него. Освен тези две операции, понякога се ползва и операцията четене от стека, което означава, че се прочита буквата на върха, без да се изважда от стека. \par

Абстрактната машина, която ще опишем, разполага с обичаен вход, от който може да чете буквите на входна азбука $X$ и стек, от който може да чете и в който може да пише. Машината може да се намира в крайно множество от състояния. \par 

\definition Недетерминиран стеков автомат наричаме седморката \\ 
\centerline{$\nsa$}
в която: $Q$ е крайно множество от състояния; $X$ е крайна входна азбука; $Z$ е крайна стекова азбука; $q_{0} \in Z$ е начална стекова буква; $\delta: Q \times (X \cup \curlies{\epsilon}) \times Z \to 2^{Q \times Z^*}$ е частична функция на преходите; $F \subseteq Q$ е множество от заключителни състояния. \par

\begin{itemize}
	\item Специалната буква $z_{0}$ в стековата азбука е предназначена да пази стека от презаписване, тъй като НСА не може да работи без да чете букви от стека. Противно на КДА четенето на входна буква от $X$ не е задължително за НСА(забележете, че $X \cup \curlies{\epsilon}$ е в дефиниционната област на $\delta$). С други думи НСА може да работи без да чете входни данни, обработвайки запомнената в стека дума.
	\item Втората особеност е по-сложната функция на преходите $\delta$, дефинирана върху тройки(състояние, входна буква или празна дума, стекова буква). В резултата на всяка стъпка на НСА може да премине недетерминирано в някое от определените от $\delta$ състояния, замествайки буквата във във върха на стека със съответната, зададена от $\delta$, дума.  
\end{itemize} \par

За формално дефиниране на работата на НСА, да определим неговото текущо положение с тройката $< q, \alpha, \gamma >$, наречена \emph{конфигурация} на НСА, в която $q$ е текущото състояние на автомата, а $\gamma$ текущата стекова дума, четена от върха към дъното на стека. Очевидно, конфигурацията точно определя текущото положение на НСА и бъдещето му поведение. \par

\definition Релацията $R_{\models} \subseteq K_{A} \times K_{A}$ дефинираме като рефлексивно и транзитивно затваряне на $R_{\vdash}$. Ако $k_{1}, k_{2} \in K_{A}$ и $k_{1} \models k_{2}$ казваме, че конфигурацията на $k_{1}$ може да се трансформира в $k_{2}$. \par

\definition НСА $\nsa$ \emph{разпознава със заключитело състояние} думата $\alpha$, ако съществува $q \in F$, такова че $< q_{0},\alpha, z_{0}> \models < q, \epsilon, \gamma >$, независимо от това каква е $\gamma$. Езикът \\
\centerline{$L_{A}^{F} = \curlies{\alpha | \exists q \in F, < q_{0}, \alpha, z_{0} > \models < q, \epsilon, \gamma>}$,}\\
наричаме език, разпознавам със заключителни състояния от НСА $A$. \par

\definition Казваме че НСА $A = < Q, X, Z, q_{0}, z_{0}, \delta, \emptyset>$ \emph{разпознава с празен стек} думата $\alpha$, ако $< q_{0}, \alpha, z_{0} > \models < q, \epsilon, \epsilon>$, независимо от това какво е състоянието $q$. Езикът \\
\centerline{$L_{A}^{\emptyset} = \curlies{\alpha | < q_{0}, \alpha, z_{0}> \models < q, \epsilon, \epsilon>}$,}\\
наричаме език, разнознат с празен стек от НСА $A$  

\subsection{Връзка между НСА и КСГ}
\theorem За всеки НСА $A$, разпознаващ езика $L$ съществува КСГ $\Gamma$ такава, че $L = L_{G}$\\

\proof 
Доказателството е взето от \path{https://store.fmi.uni-sofia.bg/fmi/logic/stefanv/files/eai/eai-notes.pdf}, стр. 63\par

Ще разгледаме двете посоки на тръврдението поотделно.

\enumNum
\begin{enumerate}
	\item Нека е дадена безконтекстна граматика $\cfgVers$. Нашата цел е да построим стеков автомат $P$, така че $L_{s}(P) = L(G)$. Нека \\
	\centerline{$P = \langle \curlies{q}, \Sigma, \Sigma \cup V, S, q, \Delta, \emptyset$}, \\
	където функцията на преходите е:\\
	$\Delta(q, \epsilon, A) = \curlies{(q, \alpha) | A \to \alpha е правило в граматиката G}$ \\
	$\Delta(q, \alpha, \alpha) = \curlies{(q, \epsilon)}$\\
	\item Нека имаме $P = \langle Q, \Sigma, \Gamma, \Delta, s, \#, \emptyset \rangle$. 
	Ще дефинираме безконтекстна граматика $G$, за която $L_{S}(P) = L(G)$. Променливите на граматиката са \\
	\centerline{$V = \curlies{[q, A, p] | q, p \in Q, A \in \Gamma}$.} \\
\end{enumerate}
Правилата на $G$ са следните:

\begin{itemize}
	\item $S \to [s, \#, q]$, за всяко $q \in Q$; \\
	\item $[q, A, q_{m+1}] \to a[q_{1}, B_{1}, q_{2}][q_{2}, B_{2}, q_{3}]...[q_{m}, B_{m}, q_{m+1}]$, където \\
	\centerline{$(q_{1}, B_{1}...B_{m}) \in \Delta(q, a, A)$}\\
	и произволни $q, q_{1}, ..., q_{m+1} \in Q, a \in \Sigma \cup \curlies{\epsilon}$. \\
	Да обърнем внимание, че е възможно $m = 0$. Това означава, че $(q_{1}, \epsilon) \in \Delta(q, a, A)$ и тогава имаме правилото $[q, A, q_{1}] \to a$, където $а \iн \Sigma \cup \curlies{\epsilon}$. \\
	Трябва да докажем, че: \\
	\centerline{$[q, A, p]\to^{*}_{G} \alpha \leftrightarrow (q, \alpha, A) \vdash^{*}_{P}(p, \epsilon, \epsilon)$}.\\
\end{itemize}
$(\rightarrow)$ С пълна индукция по $i$ ще докажем, че \\
\centerline{$(q, \alpha, A) \vdash^{i}_{P} (p, \epsilon, \epsilon) \rightarrow [q, A, p] \rightarrow^{*}_{G} \alpha$.}\\

Ако $i = 1$, то е лесно, защото $\alpha \in \Sigma \cup \curlies{\epsilon}$ и $m = 0$.\\

Ако $i > 1$, нека $\alpha = a \beta$. Тогава:\\
\centerline{$(q, a \beta,A)\vdash_{P}(q_{1}, \beta, B_{1}...B_{n})\vdash^{i-1}_{P}(p, \epsilon, \epsilon)$}.\\

Да разбием думата $\beta$ на $n$ части, $\beta = \beta_{1}...\beta_{n}$, със свойството, че след като прочетем $\beta_{i}$ сме премахнали променливата $B_{i}$ от върха на стека. Това означава, че: \\
\centerline{$(q_{j}, \beta_{j}, B_{j}) \vdash^{l_j}_{P}(q_{j+1, \epsilon, \epsilon})$, за $j = 1, ..., n-1$,}\\
\centerline{$(q_{n}, \beta_{n}, B_{n}) \vdash^{l_n}_{P}(p, \epsilon, \epsilon)$,}\\
където $l_{1} + l_{2} + ... + l_{n} = i - 1$. Сега по \textbf{И.П}(индукционно предположение?) получаваме:\\
$(q_{j}, \beta_{j}, B_{j})\vdash^{l_j}_{P}(q_{j+1}, \epsilon, \epsilon) \rightarrow [q_{j}, B_{j}, q_{j+1}] \to^{*}_{G}\beta_{j}$, за $j = 1, ..., n - 1$, \\
$(q_{n}, \beta_{n}, B_{n})\vdash^{l_n}_{P}(p, \epsilon, \epsilon) \rightarrow [q_{n}, B_{n},p] \to^{*}_{G}\beta_{n}$.\\
Обединявайки тези изводи с правилото\\
\centerline{$[q, A, p] \to_{G} a[q_{1}, B_{1}, q_{2}]...[q_{n}, B_{n}, p]$,}\\
получавам извода \\
\centerline{$[q, A, p] \to ^{*}_{G} a \beta$}\\ \\

($\leftarrow$) Отново с пълна индукция по $i$ ще докажем, че\\
\centerline{$[q, A,p] \to^{i}_{G}\alpha \rightarrow (q, \alpha, A) \vdash^{*}_{P} (p, \epsilon, \epsilon) $.}\\

Ако $i = 1$, то имаме $[q, A, p] \to \alpha$, където $\alpha = a$ или $\alpha = \epsilon$. Ако $i > 1$, то имаме, че $\alpha = a \beta$ и за някое $n$, \\
\centerline{$[q, A, p] \to_{G} a[q_{1}, B_{1}, q_{2}][q_{2}, B_{2}, q_{3}]...[q_{n}, B_{n}, p] \to^{i-1}_{G} \beta$.}\\

Отново нека $\beta = \beta_{1}...\beta_{n}$, където \\
\centerline{$[q_{j}, B_{j}, q_{j+1}] \to ^{i_j}_{G} \beta_{j}$, за $j = 1, ..., n-1$,}\\
\centerline{$[q_{n}, B_{n}, p] \to ^{i_n}_{G} \beta_{n}$,}\\

където $i_{1} + i_{2} + ... + i_{n} = i - 1$. От \textbf{И.П} получаваме, че \\
\centerline{$[q_{j}, B_{j}, q_{j+1}] \to ^{i_j}_{G} \beta_{j} \rightarrow (q_{j}, \beta_{j}, B_{j}) \vdash^{*}_{P}(q_{j+1}, \epsilon, \epsilon), j = 1, ..., n - 1$}\\
\centerline{$[q_{n}, B_{n}, p] \to ^{i_n}_{G}\beta_{n} \rightarrow (q_{n}, \beta_{n}, B_{n}) \vdash^{*}_{P}(p, \epsilon, \epsilon)$,}\\

Обединяквайки всичко което знаем, получаваме: \\
\centerline{$(q, a \beta, A)\vdash_{P}(q_{1}, \beta_{1}...\beta_{n}, B_{1}...B_{n})$}\\
\centerline{$\vdash^{*}_{P}(q_{2}, \beta_{2}...\beta_{n}, B_{2}...B_{n})$}\\
\centerline{...}\\
\centerline{$\vdash^{*}_{P}(q_{n}, \beta_{n}, B_{n})$}\\
\centerline{$\vdash^{*}_{P}(p, \epsilon, \epsilon)$}\par


\end{document} 