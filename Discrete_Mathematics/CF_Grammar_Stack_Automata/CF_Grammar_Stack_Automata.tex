% !TEX TS-program = pdflatex
% !TEX encoding = UTF-8 Unicode

% This is a simple template for a LaTeX document using the "article" class.
% See "book", "report", "letter" for other types of document.

\documentclass[11pt]{article} % use larger type; default would be 10pt

\usepackage[utf8]{inputenc} % set input encoding (not needed with XeLaTeX)

%%% Examples of Article customizations
% These packages are optional, depending whether you want the features they provide.
% See the LaTeX Companion or other references for full information.

%%% PAGE DIMENSIONS
\usepackage{geometry} % to change the page dimensions
\geometry{a4paper} % or letterpaper (US) or a5paper or....
% \geometry{margin=2in} % for example, change the margins to 2 inches all round
% \geometry{landscape} % set up the page for landscape
%   read geometry.pdf for detailed page layout information

\usepackage{graphicx} % support the \includegraphics command and options

% \usepackage[parfill]{parskip} % Activate to begin paragraphs with an empty line rather than an indent

%%% PACKAGES
\usepackage{booktabs} % for much better looking tables
\usepackage{array} % for better arrays (eg matrices) in maths
\usepackage{paralist} % very flexible & customisable lists (eg. enumerate/itemize, etc.)
\usepackage{verbatim} % adds environment for commenting out blocks of text & for better verbatim
\usepackage{subfig} % make it possible to include more than one captioned figure/table in a single float
% These packages are all incorporated in the memoir class to one degree or another...

%%% HEADERS & FOOTERS
\usepackage{fancyhdr} % This should be set AFTER setting up the page geometry
\pagestyle{fancy} % options: empty , plain , fancy
\renewcommand{\headrulewidth}{0pt} % customise the layout...
\lhead{}\chead{}\rhead{}
\lfoot{}\cfoot{\thepage}\rfoot{}

%%% SECTION TITLE APPEARANCE
\usepackage{sectsty}


\allsectionsfont{\sffamily\mdseries\upshape} % (See the fntguide.pdf for font help)
% (This matches ConTeXt defaults)

%%% ToC (table of contents) APPEARANCE
\usepackage[nottoc,notlof,notlot]{tocbibind} % Put the bibliography in the ToC
\usepackage[titles,subfigure]{tocloft} % Alter the style of the Table of Contents
\renewcommand{\cftsecfont}{\rmfamily\mdseries\upshape}
\renewcommand{\cftsecpagefont}{\rmfamily\mdseries\upshape} % No bold!

%%% END Article customizations


\usepackage[bulgarian]{babel}
\usepackage{physics}
\usepackage{amsmath}
\usepackage{centernot}
\usepackage{url}

%%% The "real" document content comes below...

\title{Контекстно свободни граматики. Стекови автомати}
\author{Play4u}
%\date{} % Activate to display a given date or no date (if empty),
         % otherwise the current date is printed 

\begin{document}
\maketitle

\newcommand{\lrangle}[1]{\left\langle #1 \right\rangle}

\newcommand{\belongsTo}{\in}
\newcommand{\notBelongsTo}{\centernot\in}

\newcommand{\kda}{A = <Q, X, q_{0}, \delta, F>}
\newcommand{\cfg}{\Gamma = <N, T, S, P>}

\newcommand{\italicBold}[1]{\textbf{\emph{#1}}}
\newcommand{\definition}{\italicBold{Дефиниция:}}
\newcommand{\theorem}{\italicBold{Теорема:}}
\newcommand{\lemma}{\italicBold{Лема:}}
\newcommand{\proof}{\italicBold{Доказателство:}}

\newcommand{\curlies}[1]{\{#1\}}

\newcommand{\enumNum}{\renewcommand{\theenumi}{\arabic{enumi}}}
\newcommand{\enumlet}{\renewcommand{\theenumi}{\alph{enumi}}}

\section{Контекстно свободни граматики}
Всеки контекстно-свободен език се поражда от контекстно-свободна граматика $\cfg$, правилата на която са от вида $А \to \alpha, A \in N, \alpha \in (N \cup T)^{+}$.
Ако в граматиката няма аксиома в дясба част на правило, допустимо е и правилото $S \to \epsilon$, позволяващо ни да извеждаме само празната дума. \par

На всеки извод в контекстно-свободната граматика ще  съпоставим дървовидна структура със следната 

\definition{Нека $\cfg$ е контекстно-свободна граматика. Дървото $D(V, E)$ с корен $r \in V$, с наредба на синовете и функция $f: V \to (N \cup T)$< съпоставяща на всеки връх буква от азбуките на граматиката, наричаме \emph{дърво на извод} в $\Gamma$, ако 

\enumlet
\begin{enumerate}
	\item За всеки нелист $v$ на $D, f(v) \in N$, като $f(r) = S$;\\
	\item За всеки лист $l$ на $D, f(l) \in T$;\\
	\item За всеки нелист $v$ с наредена последвоателност от синове $v_{i_1}, v_{i_2}, ..., v_{i_k}$, е в сила $f(v_{i_1})f(v_{i_2})...f(v_{i_k})\in P$. 
\end{enumerate}

\italicBold{Бележка от автора: В конспекта за държавния изпит се споменава въпроса "Дърво на синтаксиса", докато в учебника по ДСТР присъства единствено понятието "Дърво на изводите" (което ще се използва и в този документ). Автора предполага, че двете са едно и също чрез следните доводи:} 

\enumNum
\begin{enumerate}
	\item Предполага, че превода на "извод" на англ. е "derivation".\\
	\item Според \path{https://en.wikipedia.org/wiki/Parse_tree} Syntax tree $\equiv$ Derivation tree \\
\end{enumerate} \par

Дума на \emph{дърво на извод} дефинираме индуктивно, следвайки една от дефинициите на понятието дърво.\\

\definition Нека $D(V, E)$ е дърво на извод в граматиката $\cfg$

\enumlet
\begin{enumerate}
	\item Ако $D$ е дърво от вида
		\centerline{$S$-------$\epsilon$}
		то дума на $D$ е $\epsilon$;\\
	\item Нека $D'(\curlies{l}, \curlies{})$ е поддърво на дървото на извод $D$, състоящо се само от лист $l$. Дума на $D'$ е $f(l)$;\\
	\item Нека $D'$ е поддърво на $D$ с корен $r', D_{1}, D_{2}, ..., D_{k}$ са всички поддървета на $D'$ с корени синовете на $r'$, подредену в съотвествие с наредбата на тези синове и $\alpha_{i}$ е думата на поддървото $D_{i}, i = 1, 2, ..., k$. Тогава думата на $D'$ е $\alpha_{1}, \alpha_{2}, ...,\alpha_{k}$. 
\end{enumerate} \par

\theorem Нека $\cfg$ е КСГ. Съществува дърво на извод $D$ в $\Gamma_{A}$ с дума $\omega$ ще покажем, че $A^{\Gamma} \models^{A} \omega$ \\
\centerline \emph{Insert proof here}

\definition Ако в КСГ $\Gamma$ съществува повече от едно дърво на извод за някоя дума $\omega$, тогава наричаме $\Gamma$ \emph{еднозначна} \\ 

\definition КСЕ(език) $L$, закойто всяка КСГ, която го поражда е нееднозначна, наричаме \emph{нееднозначен език}.

\section{Нормална форма на Чомски}

\italicBold{Не успях да намеря информация в учебника по ДСТР. Всичко в този раздел  е взето от \path{https://store.fmi.uni-sofia.bg/fmi/logic/stefanv/files/eai/eai-notes.pdf} стр. 57} \\ \par

\definition Една КСГ е в нормална форма на Чомски ако всяко правило е от вида \\
\centerline{$A \to BC$ и $A \to \alpha$}
Като $B$ и $C$ не могат да бъдат променливата за начало $S$\\ 
\centerline{\emph{Insert proof here}}

\end{document} 

\section{Стекови автомати}

