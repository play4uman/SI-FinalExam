\documentclass[11pt]{article}
\usepackage[utf8]{inputenc}
\usepackage{geometry}
\geometry{a4paper}
\usepackage{graphicx}

\usepackage{booktabs}
\usepackage{array} 
\usepackage{paralist} 
\usepackage{verbatim} 
\usepackage{subfig}

\usepackage{fancyhdr} 
\pagestyle{fancy}
\renewcommand{\headrulewidth}{0pt} 
\lhead{}\chead{}\rhead{}
\lfoot{}\cfoot{\thepage}\rfoot{}

\usepackage{sectsty}

\allsectionsfont{\sffamily\mdseries\upshape}

\usepackage[nottoc,notlof,notlot]{tocbibind} 
\usepackage[titles,subfigure]{tocloft} 
\renewcommand{\cftsecfont}{\rmfamily\mdseries\upshape}
\renewcommand{\cftsecpagefont}{\rmfamily\mdseries\upshape} 

\usepackage[bulgarian]{babel}
\usepackage{physics}
\usepackage{amsmath}
\usepackage{centernot}
\usepackage{enumitem}

\title{3. Крайни автомати. Регулярни езици. Теорема на Клини}

\begin{document}
\maketitle

\textbf{\textit{Анотация:}} Детерминирани крайни автомати. Регулярни операции. Недетерминирани крайни автомати. Представяне на всеки недетерминиран краен автомат с детерминиран такъв(с доказателство). Затвореност относно регулярните операции. Теорема на Клини(с доказателство). Лема за покачването (uvw) (с доказателство). Примери за регулярни и нерегулярни езици. Минимизация на състоянията. Теорема на Майхил-Нероуд(с доказателство). Алгоритъм за конструиране на минимален автомат, еквивалентен на даден детерминиран краен автомат.
\newcommand{\lrangle}[1]{\left\langle #1 \right\rangle}

\newcommand{\belongsTo}{\in}
\newcommand{\notBelongsTo}{\centernot\in}
\newcommand{\kda}{A = <Q, X, q_{0}, \delta, F>}

\newcommand{\italicBold}[1]{\textbf{\emph{#1}}}
\newcommand{\definition}{\italicBold{Дефиниция:}}
\newcommand{\theorem}{\italicBold{Теорема:}}
\newcommand{\lemma}{\italicBold{Лема:}}
\newcommand{\proof}{\italicBold{Доказателство:}}

\newcommand{\curlies}[1]{\{#1\}}

\section{Крайни детерминирани автомати}
Нека $X = \{x_1, x_2, ..., x_n\}$ е крайна азбука, $X^*$ е множеството от всички думи над $X$, а $X^{+} = X^{k}$ - множеството от думите с дължина $k$, $X^{1} = X$, а $X^{0}$ се състои само от $\varepsilon$. \par

\italicBold{Твърдение:}\emph{Множеството $X^{*}$ от думите над крайната азбука $X$ е изброимо}. \\

Формалните езици по естествен начин дефинират езици над множеството си терминални символи. Те обаче не позволяват да се даде по един достатъчен ясен начин отговор на въпроси от вида "Даден e език чрез граматиката си $\Gamma = \left\langle N, T, S, P \right\rangle$ и дума $\alpha \epsilon \Gamma^*.$ Принадлежи ли $\alpha$ на $\Gamma?$" \par

За целта можем да използваме т.нар. абстрактни математически машини. Всяка една такава машина можем да разглеждаме като черна кутия, която чете от входа дума над дадена азбука и може да изведе на изхода дума над друга (не непременно различна от входната азбука). Характерно е, че във всеки момент от работата си машината се намира в някакво състояние q, принадлежащо на крайно множество $Q$ от състояния. Смяната на едно състояние в друго става в изброимо множество от моменти на времето, наричани тактове. Между всеки два такта машината остава в едно и също състояние. Новото състояние се определя еднозначно от текущото състояние и входната буква, която машината чете в този момент. Изходната буква, когато машината действително извежда нещо, също е функция на текущото състояние и входна буква. \par

В началото машината винаги се намира в едно и също състояние, наричано "начално състояние". Работата й се определя от цикличното повтаряне на няколко прости действия - прочитане на входна буква, определяне на от тази буква и текущото състояние на изходна буква и следващото състояние, извеждане на изходната буква и смяна на текущото състояние със следващото, изчакване до настъпване на следващият акт, когато действията се повтарят отново. \par

Абстрактната машина може да завърши работа, когато достигне някое от предварително фиксираните заключителни състояния или когато функцията, определяща следващото състояние е частична (машината не може да продължи да работи заради недефинираност). Разбира се, спирането може да се извърши и при прочитането на входна дума докрай или при някое друго, свързано с конкретната дефиниция събитие. Има машини, които са в състояние никога да не завършат работа при определени обстоятелства. \\ 

\textbf{\emph{Дефиниция:}} \emph{Краен детерминиран автомат (КДА) наричаме петорката} $A = \lrangle{Q, X, q_{0}, \delta, F}$, в която: $Q$ е крайно множество от състояния; $X$ е крайна входна азбука; $q_{0} \epsilon Q$ е начално състояние; $\delta : Q \times X \to Q$ е частична функция на преходите, пресмятаща следващото състояние; $F \subseteq Q$ са заключителните състояния на КДА. \par

Можем да представим КДА чрез краен ориентиран мултиграф, с върхове елементите на Q, в който върхът $q_{i} \epsilon Q$ и върхът $q_{j} \epsilon Q$ са свързани с ребро, надписано с $x \epsilon X$, ако $\delta(q_{i}, x) = q_{j}$.

\italicBold{Дефиниция:} Ако $(q, w)$ и $(q', w')$ са две конфигурации на автомата $M$. Тогава $(q,w) \vdash_{M} (q', w')$ е изпълнено тогава и само тогава, когато $w = aw'$ за някой символ $а \in \Sigma$ и $\delta(q,a) = q'$. Ако се намираме в конфигурацията $(q, \epsilon)$, това означава, че сме прочели цялата дума и работата на автомата е приключила (тук използваме $\epsilon$ като означение за празната дума).
Ще означаваме рефлексивната и транзитивно затворена релация на $\vdash_{M}$ с $\vdash^*_{M}$, което означава, че ако $(q, w) \vdash^{*}_{M} (q', w')$, то тогава $(q', w')$ е достижимо от $(q,w)$ след последователност от преходи или дори без нито един преход. Така можем да дефинираме и кога една дума се разпознава от автомата $M$.

\italicBold{Дефиниция:} Казваме, че КДА $\kda $ с разширена функция на преходите $\Delta$ разпознава думата $\alpha \belongsTo X^{*}$, ако $\Delta(q_{0}, \alpha) \belongsTo F$. Множеството $L_{A} = \{\alpha| \alpha \belongsTo X^{*}, \Delta(q_{0}, \alpha) \belongsTo F\}$ наричаме език, разпознаван от крайния детерминиран автомат $A$ \par
Ако функцията $\delta$ на КДА $\kda$ не е тотална, тогава можем да разширим $A$ до автомата 
$A' = < Q' = Q \cup \{q^{*}\}, X, q_{0}, \delta', F>$, такъв че 
$q^{*} \centernot\in F, \delta' : Q' \times X \to Q'$, а
$\delta'(q, x) = 
	\begin{cases}
		\delta(q,x) \quad \text{ако } \delta(q,x) 
		\text{ е дефинирана}\\
		q^{*} \quad \quad \quad \text{ако } \delta(q,x)
		\text{ не е дефинирана} \\
	\end{cases}$\par
Всички думи, разпознавани от $А$ се разпознават и от $A'$. Всяка дума $\alpha$, за която 
$\Delta_{A}(q_{0},\alpha) \notBelongsTo F$, и значи не се разпознава от $A$ - не се разпознава и от $A'$, а за всяка дума $\beta$, за която 
$\Delta_{A}(q_{0}, \beta) не е дефинирана$ в $A'$ имаме
$\Delta_{A'}(q_{0}, \beta) = q^{*} \notBelongsTo F$ и $\beta$ отново не се разпознава. \\
Така $L_{A} = L_{A'}$ и следователно можем да работим и с недодефинирани автомати или да ги додефинираме (без да изменяме езика им), когато това е необходимо. \par
Сега ще покажем каква е връзката между КДА и автоматните езици\\
\italicBold{Теорема:} За всеки КДА, $\kda$ съществува автоматна граматика $\Gamma$ такава, че $L_{\Gamma} = L_{A}$

\section{Крайни недетерминирани автомати}
\italicBold{Дефиниция:} Петорката $\kda$, където $Q, X, q_{0}$ и $F$ са дефинирани както при КДА, а функцията на преходите е $\delta : Q \times X \to 2^{Q}$, наричаме краен недетерминиран автомат (КНА) \\

КНА работи по следния начин: Когато в резултат на изчисление на функцията $\delta$ се получи множеството $Q'$ от състояния в които КНА трябва да премине, той се размножава в $|Q'|$ копия и всяко копие преминава в едно от състоянията на $Q'$. \textbf{Можем да си мислим, че автоматът едновременно се намира във всичките състояния на $Q'$.} Когато КНА прочете следващата буква, всяко от копията се размножава в толкова нови копия, колкото текущото му състояние, входната буква и функцията $\delta$ определят и всяко едно копие преминава в съответното състояние. Множеството от състояния в които се намира КНА ще съдържа състоянията на всички получени копия. Тази дефиниция ще уточним формално по следния начин.\\

\italicBold{Дефиниция:} Казваме, че КНА $A$ разпознава думата $\alpha \belongsTo X^{*}$, ако $\Delta(q_{0}, \alpha) \cap F \neq \emptyset$. Език, разпознаван от КНА $А$ определяме като $L_{A} = \{\alpha|\alpha \belongsTo X^{*}, \Delta(q_{0}, \alpha)\cap F \neq \emptyset \}$\\

Всяка една наредена тройка $(q, u, p) \in \Delta$ се нарича преход в $А$ от $q$ в $p$ при входна буква $u$. Възможно е да имаме преходи от вида $(q, \epsilon, p)$, при които от $q$ се преминава в $p$ без да се чете буква от входа. Релацията $\vdash^{*}_{A}$, се дефинира по аналогичен начин на тази в КДА.\\

\theorem \emph{ За всяка автоматна граматика $\Gamma$ съществува КНА $A$, такъв че $L_{\Gamma} = L_{A}$.}\\

\italicBold{Дефиниция:}  Два автомата $M_{1}$ и $M_{2}$ са еквивалентни $\iff$ $L(M_{1}) = L(M_{2})$\\

Вече можем да въведем и следната теорема\\
\italicBold{Теорема:} За всеки недетерминиран автомат, съществува еквивалентен детерминиран автомат.\\
\italicBold{Доказателство:} Нека $\kda$. Построяваме КДА $A' = < Q', X, t_{0}, \delta', F'$, където $Q' \subseteq 2^{Q}$. Нека множеството $\{q_{p1}, q_{p2}, ..., q_{pl} \} \belongsTo Q'$. За по кратко ще го означаваме с $t_{[p1, p2,..., pl ]}$. При това означение определяме 
$t_{0} = \{q_{0}\} = t_{|0|}.$. Нека $F' = \{t_{[p1, p2, ..., pl]}|\{ q_{p1}, q_{p2}, ..., q_{pl} \} \cap F \neq \emptyset\}, a \delta'(t_{[p1, p2, ... ,pl]}, x) = t_{[r1, r2, ..., rm]}, \text{ ако } \{q_{r1}, q_{r2}, ..., q_{rm}\} = \cup_{i = 1}^{l}\delta(q_{pi}, x)$. Забележете, че не можем да фиксираме в явен вид кои точно подмножества на $Q$ влизат в $Q'$. Те се определят от изчисляването на функцията $\delta'$, започвайки от $\delta'(t_{|0|}, \alpha), \forall x \in X$. С индукция по дължината на $\alpha$ ще покажем, че $\Delta_{A'}(t_{|0|}, \alpha) = t_{[p1, p2, ..., pl]} \Leftrightarrow 
\Delta_{A}(q_{0}, \alpha) = \{q_{p1}, q_{p2}, ..., q_{pl}\}$.

\renewcommand{\theenumi}{\alph{enumi}}
\begin{enumerate}
	\item Нека $\omega = \epsilon$. Тогава $\Delta_{A'}(t_{|0|}, \epsilon) = t_{|0|}, a \Delta_{A}(q_{0}, \epsilon)= \{q_{0}\}$ и твърдението е в сила
	\item Допускаме верността на твърдението за някоя дума $\alpha$  и ще покажем, че $\Delta_{A'}(t_{t|0|}, \alpha x) = t_{[r1, r2, ..., rm]} \leftrightarrow \Delta_{A}(q_{0}, \alpha x) = 			\{q_{r1}, q_{r2}, ..., q_{rm}\}$.
	\item Нека $\Delta_{A'}(t_{|0|}, \alpha x) = t_{[r1, r2, ..., rm]}$, като
		$\Delta_{A'}(t_{|0|}, \alpha) = t_{[p1, p2, ..., pl]}$. Тогава $\delta'(t_{[p1, p2, ..., pl]}, x) = t_{[r1, r2, ...,rm]}$ и съгласно построението на $\delta'\{q_{r1}, q_{r2}, 				..., q_{rm} \} = $
		$\cup_{i=1}^{l}\delta(q_{pi}, x).$ Но от индуктивното допускане 
		$\Delta_{A}(q_{0}, \alpha) = \{q_{p1}, q_{p2}, ..., q_{pl}\} $ и следователно 
		$\Delta_{A}(q_{0}, \alpha x) = \{q_{r1}, q_{r2}, ..., q_{rm} \}$. Разсъжденията в другата посока са подобни.
\end{enumerate} \par

Като вземем предвид дефиницията на $F'$ и току що доказаното, получаваме $\Delta_{A'}(t_{|0|}, \alpha) \in F' \leftrightarrow 
\Delta_{A}(q_{0}, \alpha) \cap F \neq \emptyset$ и значи 
$L_{A'} = L_{A}$   

\section{Затвореност относно регулярните операции}
\begin{itemize}[noitemsep]
	\item Сечение: Ако $L_{1}$ и $L_{2}$ са два произволни автоматни езика над азбуката $\Sigma$, то $L_{1}\cap L_{2}$ също е автоматен език.
	\item Допълнение: Ако $L$ е автоматен език, то $\Sigma^{*}\setminus L$ също е автоматен.
	\item Обединение: Ако $L_{1}$ и $L_{2}$ са два произволни автоматни езика, то $L_{1}\cup L_{2}$ също е автоматен език.
	\item Конкатенация: Ако $L_{1}$ и $L_{2}$ са два произволни автоматни езика, то $L_{1}. L_{2}$ също е автоматен език.
	\item Звезда на Клини: Ако $L$ е произволен автоматен език, то и $L^{*}$ е автоматен език
\end{itemize}

\section{Регулярни езици}
\italicBold{Определение:} Нека е дадена азбука $\Sigma$. Дефинираме множеството от \emph{регулярни езици} над азбуката $\Sigma$ и едновременно с това множеството от регулярни \emph{регулярни изрази}, които разпознават тези езици. 
\renewcommand{\theenumi}{\arabic{enumi}}
\begin{enumerate}
	\item За всеки символ $a \in \Sigma, \curlies{{a}}$ е регулярен език, който се разпознава от регулярния израз $a$;
	\item $\curlies{{\epsilon}}$ е регулярен език, който се разпознава от регулярния израз $\epsilon$;
	\item $\emptyset$ е регулярен език, който се разпознава от регулярния израз $\emptyset$;
	\item $L_{1}\cup L_{2}$, където $L_{1}$ и $L_{2}$ са регулярни езици, който се разпознава от регулярния израз $(r_{1} + r_{2})$, където $r_{1}$ и $r_{2}$ са регулярните изрази за $L_{1}$ и $L_{2}$. Записваме, че $\mathcal{L}(r_{1})\cup \mathcal{L}(r_{2}) = \mathcal{L}(r_{1} + r_{2})$;
	\item $L_{1}.L_{2} = \curlies{{uw | u \in L_{1} \And w \in L_{2}}}$, където $L_{1}$ и $L_{2}$, който се разпознава от регулярния израз $(r_{1} . r_{2})$, където $r_{1}$ и $r_{2}$ са регулярните изрази за $L_{1}$ и $L_{2}$. Записваме, че $\mathcal{L}(r_{1}). \mathcal{L}(r_{2}) = \mathcal{L}(r_{1} . r_{2})$;
	\item $L^{*} = \curlies{{w_{1}w_{2}...w_{n}|n \in N \And w_{i} \in L}, i \leq n}$, където $L$ е регулярен език, който се разпознава от регулярния израз $(r^{*})$, където $r$ е регулярния израз за $L$. Записваме, че $\mathcal{L}(r)^{*} = \mathcal{L}(r^{*})$;
\end{enumerate} \par
\subsection{Примери за регулярни езици}
$\curlies{w \in \curlies{a,b}^{*}}$ всяко $a$ да е веднага последвано от $b$ е регулярен език.\\

$ \begin{array}{ll}%
	\text{Израз}  &  \text{Език}  \\
	a & \curlies{a} \\
	a + b & \curlies{a,b} \\
	a(a + b) & \curlies{aa, ab} \\
	a^{*} & \curlies{\epsilon, a, a^{2}, ..., a^{i}, ...} \\
	a^{*}(a + b) & \curlies{a, b, aa, ab, a^{2}a, a^{2}b, ..., a^{i} a, a^{i}b, ...} \\
	(a + b)^{*} & X^{*} \\                                           
	a^{*}b + b^{*}a & \curlies{b, a, ab, ba, a^{2}b, b^{2}a, ..., 	 a^{i}b, b^{i}a, ...}
\end{array}$%
\subsection{Примери за нерегулярни езици}
$L = \curlies{{a^{n}b^{n} | n \in N}}$\\
$L = \curlies{{a^{m}b^{n} | n, m \in N \And m < n}}$\\
$L = \curlies{{a^{n} | n \textit{ е просто число}}}$\\

\subsection{Теорема на Клини}
\emph{Множествата на регулярните и автоматните езици съвпадат (т.е. всеки език разпознават от краен автомат е регулярен)}\\
\italicBold{Доказателство:}
\renewcommand{\theenumi}{\arabic{enumi}}
\begin{enumerate}
	\item Ще покажем s индукция по дефиницията на регулярни езици, че всеки регулярен език е автоматен. Действително $\curlies{\epsilon, \emptyset} \text{ и } \curlies{x_{i}, i 				= 1,2,...,n}$ са автоматни езици, защото са крайни. Ако 				допуснем, че регулярните езици $L_{\alpha}$ и $L_{\beta}$, 				съответни на регулярните изрази $\alpha$ и $\beta$ са 					автоматни тогава
		$L_{\alpha} + L_{\beta}, L_{\alpha}.L_{\beta}$ и $L_{\alpha} ^{*}$ са автоматни, защото са сума, произведение и итерация на автоматни езици, съответно. Тъй като други регулярни 				езици няма, всеки регулярен език е автоматичен. 
	\item Нека езикът $L$ е автоматен. Ще докажем, че $L$ е регулярен език. Съществува КДА $\kda$ такъв, че $L = L_{A}$. Нека автоматът $A$ e представен с крайния ориентиран 					мултиграф $G$. Нека състоянията на $А$ са $Q = 							\curlies{q_{0}, q_{1}, ..., q_{n}}$ и $F = \curlies{q_{p1}, 			q_{p2}, ..., q_{pr}}$. 
		Да означим с $R_{ij}^{k}$ множеството от маршрутите в $G$ от връх $q_{i}$ до връх $q_{j}$, които не използват като вътрешни върхове $q_{k}, q_{k+1}, ..., q_{n}$.
		Очевидно, всеки маршрут от $q_{i}$ до $q_{j}$ означава определена дума $\alpha \in X^{*}$, такава че $\Delta(q_{i}, \alpha) = q_{j}$. Така на множеството от маршрути $R_{ij}				^{k}$ можем да гледаме като на множество от съответните думи 		от $X^{*}$, т.е. $R_{ij}^{k}$ e език над $X^{*}$ - маршрутен 		език. В автомата няма състояния с номера по-големи от $n$ и 			затова $R_{ij}^{k}$ е езикът на всички маршрути от $q_{i}$ 				до $q_{j}$, когато $k > n$.
\end{enumerate}
\subsection{Минимизация на КДА}
Както стана ясно по-горе, два крайни детерминирани автомата са еквивалентни, езиците ми съвпадат. Естествено е желанието, когато език се разпознава от повече от един автомат, да се опитаме да намерим този, който е най-прост.\\\\
\italicBold{Дефиниция:} КДА $A_{0}$, разпознаващ автоматния език $L$, наричаме \emph{минимален} за езика $L$, ако за всеки друг автомат $A$, разпознаващ $L$, $|Q_{0}| \leq |Q|$, където $Q$ и $Q_{0}$ са множествата от състояния на $A_{0}$ и $A$ съответно. \\

\textbf{Алгоритъм за конструиране на минимален детерминиран автомат еквивалентен на даден детерминиран автомат:}\\
Нека $M = <K, \Sigma, \delta, s, F>$ е детерминирам краен автомат. Дефинираме релацията $А_{M} \subseteq K x \Sigma^{*}$, като $(q, w) \in А_{M} \iff (q, w) \vdash^{*}_{M} (f, e), f \in F$. Казваме, че двете състояния $q, p \in K$ са еквивалентни (означаваме с $q \equiv p$), ако $\forall z \in \Sigma^{*} : (q, z) \in A_{M} \iff (p, z) \in A_{M}$. Класовете на еквивалентност на $\equiv$ са тези множества от състояния, които дефинират състоянията на минималния автомат, който разпознава $L(M)$. За да намерим тези класове на еквивалентност, ще пресметнем последователно класовете на еквивалентност на $\equiv_{0}, \equiv_{1}, \equiv_{2}...$, където $p \equiv_{n} q : (q, z) \in A_{M} \iff (p, z) \in A_{M}, z \in \Sigma^{*}, |z| \leq n$. Това ще направим последователно, като намираме класовете на еквивалентност на една релация от класовете на еквивалентност на предишната релация.
\renewcommand{\theenumi}{\arabic{enumi}}
\begin{enumerate}
	\item За $\equiv_{0}$ е очевидно, че класовете на еквивалентност са $F$ и $K-F$.
	\item Да предположим, че сме намерили класовете на еквивалентност на $\equiv_{n}$. Тогава всеки две състояния $q, p \in K, p \equiv_{n+1} q \iff p \equiv_{n} q$ и $\forall a \in \Sigma, \delta(q, a) \equiv_{n} \delta(p, a)$. По този начин можем да разберем дали две състояния са в един клас на еквивалентност или не. От това можем да намерим класовете на еквивалентност на $\equiv_{n}$.
	\item Продължаваме докато не получим, че класовете на еквивалентност на $\equiv_{n}$ са същите като класовете на еквивалентност на $\equiv_{n-1}$
\end{enumerate}
Алгоритъмът ще завърши след краен брой стъпки, тъй като на всяка една стъпка класовете на еквивалентност се увеличават поне с 1 и не може да има повече класове на еквивалентност от колкото състояния има в $M$.

\subsection{Теорема на Майхил-Нероуд}
\italicBold{Дефиниция:} Нека $L \subseteq \Sigma^{*}$ е език и нека $x, y \in \Sigma^{*}$. Казваме, че $x$ и $y$ са еквивалентни спрямо $L (x \approx_{L} y)$, ако за всяко $z \in \Sigma^{*}$ е изпълнено следното $xz \in L \iff yz \in L$. Забележете, че $\approx_{L}$ е релация на еквивалентност. \\

\italicBold{Дефиниция:} Нека $M = <K, \Sigma, \delta, s, F>$ е детерминиран краен автомат. Казваме, че две думи $x, y \in \Sigma^{*}$, са еквивалентни спрямо $М (x \sim_{M} y)$, ако има състояние $q$, такова, че $(s, x) \vdash^{*}_{M} (q, \epsilon)$ и $(s, y) \vdash^{*}_{M} (q, \epsilon)$. $\sim_{M}$ също е релация на еквивалентност и нейните класове на еквивалентност могат да бъдат достигнати от $s$, чрез някаква дума.\\ 

\theorem \emph{\textbf{Теорема} За всеки детерминиран краен автомат $M = <K, \Sigma, \delta, s, F>$ и две думи $x, y \in \Sigma^{*}$, ако $М (x \sim_{M} y)$, то $x \approx_{L} y$.}\\

От горната теорема следва, че всеки един клас на еквивалентност спрямо $\sim$ се съдържа в клас на еквивалентност на $\approx$ и всеки един клас на еквивалентност на $\approx$ e обединение на един или няколко класа на еквивалентност на $\sim$. Тъй като класовете на еквивалентност на $\sim$ са броя състояния на автомат разпознаващ $L(M)$, то можем да изкажем едно много важно свойство на всеки един автомат, който разпознава $L(M)$ - всеки един автомат, който разпознава $L(M)$ трябва да има поне толкова състояния, колкото класове на еквивалентност има релацията $\approx$. Следващата теорема доказва, че е възможно да се построи автомат разпознаващ $L(M)$ с точно толкова състояния, което ще означава, че това е автомата с минимален брой състояния за езика $L(M)$.\\

\theorem \emph{\textbf{Теорема на Майхил-Нероуд} Нека $L \subseteq \Sigma^{*}$ е регулярен език. Тогава съществува краен детерминиран автомат с брой състояния равен на броя на класове на еквивалентност на $\approx_{L(M)}$, който разпознава $L$. Това е автомата с минимален брой състояния, който разпознава  $L(M)$ и този автомат е единствен с точност до изоморфизъм. }\\\\
\italicBold{Доказателство:} Нека означим класа на еквивалентност спрямо $\approx_{L(M)}$, породен от $x$, чрез $[x]$. Ще построим краен детерминира автомат $M = <K, \Sigma, \delta, s, F>$, за който $L = L(M)$. Дефинираме $M$ по следния начин:
\begin{enumerate}
	\item $К = \curlies{{[x]:x \in \Sigma^{*}}}$, т.е. това са класовете на еквивалентност на $\approx_{L(M)}$.
	\item $s = [\epsilon]$, класа на еквивалентност на празната дума.
	\item $F = \curlies{{[x]:x\in L}}$
	\item Накрая дефинираме за всяко състояние $[x] \in K$ и всяка буква $a \in \Sigma, \delta([x], a) = [xa]$.
\end{enumerate}
Така дефиниран $M$ има краен брой състояния, защото $L$  е регулярен, т.е. има краен детерминиран автомат $M'$, който го разпознава. От предишната теорема знаем, че има по-малко или равен брой класове на еквивалентност в $\approx_{L}$ от колкото в $\sim_{M'}$, a $\sim_{M'}$ има краен брой класове на еквивалентност тъй като $M'$ има краен брой състояния, от където следва че и $\approx_{L}$ има краен брой класове на еквивалентност , т.е. $K$ е крайно. Освен това $\delta$ е добре дефинирана функция на преходите, тъй като е еднозначна и дефинирана за всяко едно състояние на $M$ и всяка една буква от азбуката.\\
Остава да покажем, че $L = L(M)$. Първо ще покажем, че за всеки $x, y \in \Sigma^{*}$ имаме, че $([x], y)\vdash^{*}_{M}([xy],\epsilon)$ (1). Това се доказва с индукция по $|y|$. За $y = \epsilon$ се доказва тривиално. Ако предположим, че в вярно за всяка една дължина на $y$ до $n$ и $y = y'a$, тогава от индукционното предположение имаме $([x],y'a)\vdash^{*}_{M}([xy'],a)\vdash_{M}([xy],\epsilon)$. Използвайки (1), доказателството е директно - за всяко $x \in \Sigma^{*}$, имаме че $x \in L(M) \iff ([\epsilon],x) \vdash^{*}_{M} (q, \epsilon), q \in F$, за което от (1) е същото като да кажем, че $[x] \in F$ или от дефиницията на $F, x \in L$.\\

\italicBold{Следствие:} Един език $L$ е регулярен, тогава и само тогава, когато релацията $\approx_{L}$ има краен брой класове на еквивалентност.

\section{uvw - теорема}
Спомага за показването на ограничеността на автоматните езици  и позволява да откриваме езици които не са автоматни\\
\theorem \emph{\textbf{(uvw-Теорема)}За всеки непразен автоматен език $L$ съществува цяло положително число n такова, че ако $\alpha \in L$ и $d(\alpha) > n$, то $\alpha = uvw$ и:}\\   
\renewcommand{\theenumi}{\arabic{enumi}}

\begin{enumerate}
	\item $d(v) \geq 1$
	\item $d(uv) \leq n$
	\item $\forall i = 0, 1, 2, ... $ думата $uv^{i}w \in L$
\end{enumerate}
\proof \\
За автоматния език $L$ съществува КДА $\kda$, който го разпознава. Избираме $n = |Q|$. Ще покажем, че това е търсеното цяло число. Нека $\alpha$ е дума от езика $L, \alpha = x_{i_{1}}, x_{i_{2}}, ..., x_{i_{k}}, d(\alpha) = k > n$. Тъй като разпознава $\alpha$, то $\Delta(q_{0}, \alpha) \in F$, т.е \\
\centerline{$\delta(q_{0}, x_{i_{1}}) = q_{i_{1}}, \delta(q_{i_{1}}, x_{i_{2}}) = q_{i_{2}}, ..., \delta(q_{i_{k-1}}, x_{i_{k}}) = q_{i_{k}} \in F$}. \par

Ако разгледаме редицата от състояния $q_{0}, q_{i_{1}},...,q_{i_{k}}, k > n$, през които автоматът преминава при работа върху думата $\alpha$, то от Принципа на Дирихле получаваме, че в редицата има поне една двойка съвпадащи състояния. Да изберем най-ляво разположената двойка $q_{i_{m}} = q_{i_{l}}, m < l$, т.е вляво от $q_{i_{l}}$ няма друга такава двойка. Разбиваме $\alpha$ на три поддуми $u, v$ и $w$, такива че $\Delta(q_{0}, u) = q_{i_{m}}, \Delta(q_{i_{m}}, v) = q_{i_{l}}, \Delta(q_{i_{l}}, w) = q_{i_{k}} \in F$.
Ще покажем, че тези три думи удовлетворяват твърденията на теоремата.

\renewcommand{\theenumi}{\arabic{enumi}}
\begin{enumerate}
	\item Тъй като $m \neq l, d(v) \geq 1$. 
	\item Тъй като $q_{i_m}, q_{i_l}$ е най-ляво разположената двойка $d(uv) \leq n$. В противен случай отново щяхме да можем да приложим Принципа на Дирихле и да намерим друга двойка съвпадащи състояния в редицата $q_{0}, q_{i_{1},...,q_{i_l-1}}$.
	\item С индукция по $i$ доказваме, че $\Delta(q_{0}, uv^{i}) = q_{i_m} = q_{i_l}$. Действително $\Delta(q_{0}, uv^{0}) = \Delta(q_{0}, u) = q_{i_m} = q_{i_l}$.\\
		Допускаме, че твърдението е вярно за $i$ и да разгледаме $\Delta(q_{0}, uv^{i+1}) = \Delta(\Delta(q_{0}, uv^{i}), v) = \Delta(q_{i_m}, v) = q_{i_l} = q_{i_m}$\\
		Сега за всяко $i = 0, 1, 2, ...$ имаме $\Delta(q_{0}, uv^{i}w) = \Delta(\Delta(q_{0}, uv^{i}), w) = \Delta(q_{i_l}, w) = q_{i_k} \in F$ \\
		И следователно $uv^{i}w \in L, \forall i = 0,1,2,...$
\end{enumerate}   

\section{КНА до КДА}
Нека имаме КНА $N = <Q, \Sigma, q^{0}, \delta, F>$, който разпознава език $L$. Тогава КДА $A = <Q', \Sigma, q_{0}, \delta', F'>$ може да бъде създаден, така че да разпознава $L$ по следния начин: 
\renewcommand{\theenumi}{\arabic{enumi}}
\begin{enumerate}
	\item Първоначално $Q' = \emptyset$
	\item Добавяме $q_{0}$ към $Q'$
	\item За всяко състояние в $Q'$, намираме възможното множество от състояния за всяка една входна буква използвайки $\delta$ на $N$. Ако това множество от състояния не е в $Q'$, го добавяме в $Q'$
	\item Терминалното състояние на КДА $А$ ще бъде всички състояния които съдържат $F$ (Терминалните състояния на КНА).
\end{enumerate}
\end{document}
