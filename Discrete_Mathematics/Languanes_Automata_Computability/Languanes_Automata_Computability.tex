% !TEX TS-program = pdflatex
% !TEX encoding = UTF-8 Unicode

% This is a simple template for a LaTeX document using the "article" class.
% See "book", "report", "letter" for other types of document.

\documentclass[11pt]{article} % use larger type; default would be 10pt

\usepackage[utf8]{inputenc} % set input encoding (not needed with XeLaTeX)

%%% Examples of Article customizations
% These packages are optional, depending whether you want the features they provide.
% See the LaTeX Companion or other references for full information.

%%% PAGE DIMENSIONS
\usepackage{geometry} % to change the page dimensions
\geometry{a4paper} % or letterpaper (US) or a5paper or....
% \geometry{margin=2in} % for example, change the margins to 2 inches all round
% \geometry{landscape} % set up the page for landscape
%   read geometry.pdf for detailed page layout information

\usepackage{graphicx} % support the \includegraphics command and options

% \usepackage[parfill]{parskip} % Activate to begin paragraphs with an empty line rather than an indent

%%% PACKAGES
\usepackage{booktabs} % for much better looking tables
\usepackage{array} % for better arrays (eg matrices) in maths
\usepackage{paralist} % very flexible & customisable lists (eg. enumerate/itemize, etc.)
\usepackage{verbatim} % adds environment for commenting out blocks of text & for better verbatim
\usepackage{subfig} % make it possible to include more than one captioned figure/table in a single float
% These packages are all incorporated in the memoir class to one degree or another...

%%% HEADERS & FOOTERS
\usepackage{fancyhdr} % This should be set AFTER setting up the page geometry
\pagestyle{fancy} % options: empty , plain , fancy
\renewcommand{\headrulewidth}{0pt} % customise the layout...
\lhead{}\chead{}\rhead{}
\lfoot{}\cfoot{\thepage}\rfoot{}

%%% SECTION TITLE APPEARANCE
\usepackage{sectsty}


\allsectionsfont{\sffamily\mdseries\upshape} % (See the fntguide.pdf for font help)
% (This matches ConTeXt defaults)

%%% ToC (table of contents) APPEARANCE
\usepackage[nottoc,notlof,notlot]{tocbibind} % Put the bibliography in the ToC
\usepackage[titles,subfigure]{tocloft} % Alter the style of the Table of Contents
\renewcommand{\cftsecfont}{\rmfamily\mdseries\upshape}
\renewcommand{\cftsecpagefont}{\rmfamily\mdseries\upshape} % No bold!

%%% END Article customizations


\usepackage[bulgarian]{babel}
\usepackage{physics}
\usepackage{amsmath}
\usepackage{centernot}

%%% The "real" document content comes below...

\title{Езици, автомати, изчислимост}
\author{Play4u}
%\date{} % Activate to display a given date or no date (if empty),
         % otherwise the current date is printed 

\begin{document}
\maketitle

\newcommand{\lrangle}[1]{\left\langle #1 \right\rangle}

\newcommand{\belongsTo}{\in}
\newcommand{\notBelongsTo}{\centernot\in}
\newcommand{\kda}{A = <Q, X, q_{0}, \delta, F>}

\newcommand{\italicBold}[1]{\textbf{\emph{#1}}}
\newcommand{\definition}{\italicBold{Дефиниция:}}
\newcommand{\theorem}{\italicBold{Теорема:}}
\newcommand{\lemma}{\italicBold{Лема:}}
\newcommand{\proof}{\italicBold{Доказателство:}}

\section{Формални езици}

Нека $X = \{x_1, x_2, ..., x_n\}$ е крайна азбука, $X^*$ е множеството от всички думи над $X$, $X_{*}$ е множеството от всички думи над $X$, а $X^{+} = X^{k}$ 
- множеството от думите с дължина 
$k$, $X^{1} = X$, а $X^{0}$ 
се състои само от $\varepsilon$. \par

Всяко от множествата $X^{k}, k = 0, 1, 2 ...$ е крайно $\abs{X^{k}} = \abs{X}^{k}$ и тъй като $X^{*} = X^{0} \cup X^{1} \cup X^{2}...$ е обединение на изброима фамилия изброими множества можем да приложим Теорема 1.3.1 (need citation) и да получим следното \par

\textbf{Следствие 4.1.1}: \emph{Множеството $X^{*}$ от думите над крайната азбука $X$ е изброимо} \par
   

\section{Крайни детерминирани автомати}
Формалните езици по естествен начин дефинират езици над множеството си терминални символи. Те обаче не позволяват да се даде по един достатъчен ясен начин отговор на въпроси от вида "Даден e език чрез граматиката си $\Gamma = \left\langle N, T, S, P \right\rangle$"
и дума $\alpha \epsilon \Gamma^*.$ Принадлежи ли $\alpha$ на $\Gamma?$" \par

За целта можем да използваме т.нар. абстрактни математически машини. Всяка една такава машина можем да разглеждаме като черна кутия, която чете от входа дума над дадена азбука и може да изведе на изхода дума над друга (не непременно различна от входната азбука). Характерно е, че във всеки момент от работата си машината се намира в някакво състояние q, принадлежащо ня крайно множество $Q$ от състояния. Смяната на едно състояние в друго става в изброимо множество от моменти на времето, наричани тактове. Между всеки два такта машината остава в едно и също състояние. Новото състояние се определя еднозначно от текущото състояние и входната буква, която машината чете в този момент. Изходната буква, когато машината действително извежда нещо, също е функция на текущото състояние и входна буква. \par

В началото машината винаги се намира в едно и също състояние, наричано "начално състояние". Работата й се определя от цикличното повтаряне на няколко прости действия - прочитане на входна буква, определяне на от тази буква и текущото състояние на изходна буква и следващото състояние, извеждане на изходната буква и смяна на текущото състояние със следващото, изчакване до настъпване на следващият акт, когато действията се повтарят отново. \par

Абстрактната машина може да завърши работа, кгоато достигне някое от предварително фиксираните заключителни състояния или когато функцията, определяща следващото състояние е частична (машината не може да продължи да работи заради недефинираност). Разбира се, спирането може да се извърши и при проичтането на входна дума докрай или при някое друго, свързано с конкретната дефиниция събитие. Има машини, които са в състояние никога да не завършат работа при определени обстоятелства. \par 

\textbf{\emph{Дефиниция:}} \emph{Краен детерминиран автомат (КДА) наричаме петорката} \\
\centerline{$A = \lrangle{Q, X, q_{0}, \delta, F}$}\\
в която: $Q$ е крайно множество от състояния; $X$ е крайна входна азбука; $q_{0} \epsilon Q$ е начално състояние; 
$\delta : Q \times X \to Q$ е частична функция на прходите, пресмятаща следващото състояние; $F \subseteq Q$ са заключителните състояния на КДА. \par

Можем да представим КДА чрез краен ориентиран мултиграф, с върхове елементите на Q, в който върхът $q_{i} \epsilon Q$ и върхът $q_{j} \epsilon Q$ са свързани с ребро, надписано с $x \epsilon X$, ако $\delta(q_{i}, x) = q_{j}$. \\

\centerline{\emph{Insert graph picture here}}  
\italicBold{Дефиниция:} Определяме разширена функция на преходите $\Delta : Q \times X^{*} \to Q$ по следния начин: \\

$\Delta(q, \epsilon) = q \quad \forall q \; \epsilon \; Q$ \\
$\Delta(q, \epsilon) = 
	\begin{cases}
		\delta(q, x) \quad \quad \forall q, x, \text{за които } \delta \text{е дефинирана.}\\
		недефинирана ако \delta(q, x) \quad \quad \text{Не е дефинирана}   
    \end{cases}$\\
$\Delta(q, \alpha x) = 
	\begin{cases}
		\Delta(\Delta(q, \alpha), x) \quad \forall q, \alpha, x, \text{ за които } \delta \text{ е дефинирана} \\
		\text{недефинирана} \quad \text{ако } \Delta(\Delta(q, \alpha), x) \text{ не е дефинирана} \\   
	\end{cases}$ \par

Стойността $\Delta(q, \alpha)$ е състоянието, до което ще достигне автоматът $A$, ако започне работа в състояние $q$ и прочете входна дума $\alpha$\\

\italicBold{Лема:} \emph{Нека $\Delta$ е разширената функция на преходите на КДА А 
$ = < Q, X, q_{0}, \delta, F >, a, \alpha_{1}, \alpha_{2} \; \epsilon \; X^{*}$. Тогава $\forall q \; \epsilon \; Q$ е в сила 
$\Delta(q, \alpha_{1}, \alpha_{2}) = \Delta(\Delta(q, \alpha_{1}), \alpha_{2})$.} \\

\centerline{\emph{Insert proof here}}

\italicBold{Дефиниция:} Казваме, че КДА 
$\kda $ с разширена функция на преходите 
$\Delta$ разпознава думата 
$\alpha \belongsTo X^{*}$, ако $\Delta(q_{0}, \alpha) \belongsTo F$. Множеството 
$L_{A} = \{\alpha| \alpha \belongsTo X^{*}, \Delta(q_{0}, \alpha) \belongsTo F\}$ наричнаме език, разпознаван от крайния детерминиран автомат $A$ \par

Ако функцията $\delta$ на КДА $\kda$ не е тотална, тогава можем да разширим $A$ до автомата 
$A' = < Q' = Q \cup \{q^{*}\}, X, q_{0}, \delta', F>$, такъв че 
$q^{*} \centernot\in F, \delta' : Q' \times X \to Q'$, а
$\delta'(q, x) = 
	\begin{cases}
		\delta(q,x) \quad \text{ако } \delta(q,x) 
		\text{ е дефинирана}\\
		q^{*} \quad \quad \quad \text{ако } \delta(q,x)
		\text{ не е дефинирана} \\
	\end{cases}$\par
Всички думи, разнознавани от $А$ се разпознават и от $A'$. Всяка дума $\alpha$, за която 
$\Delta_{A}(q_{0},\alpha) \notBelongsTo F$, и значи не се разпознава от $A$ - не се разпонзава и от $A'$, а за всяка дума $\beta$, за която 
$\Delta_{A}(q_{0}, \beta) не е дефинирана$ в $A'$ имаме
$\Delta_{A'}(q_{0}, \beta) = q^{*} \notBelongsTo F$ и $\beta$ отново не се разпознава. \\
Така $L_{A} = L_{A'}$ и следователно можем да работим и с недодефинирани автомати или да ги додефинираме (без да изменяме езика им), когато това е необходимо. /par
Сега ще покажем каква е връзката между КДА и автоматните езицию\\
\italicBold{Теорема:} За всеки КДА, $\kda$ съществува авоматна граматика $\Gamma$ такава, че $L_{\Gamma} = L_{A}$\\

\centerline{\emph{Insert proof here}} \par

Конструкцията на теоремата не може да се обърне в обратната посока по следната причина: Нека автоматна граматика съдържа правилата $A \to xB, A \to xC, A \to xD$ и т.н. Обратната конструкция не може еднозначно да да определи стойността на функцията $\delta(A, x)$, защото в граматиката има повече от един нетерминал, кандидат за това значение. Това ни води до идеята за следната, по-абстрактна математическа машина

\italicBold{Дефиниция:} Петорката $\kda$, където $Q, X, q_{0}$ и $F$ са дефинирани както при КДА, а функцията на преходите е $\delta : Q \times X \to 2^{Q}$, наричаме краен недетерминиран автомат (КНА) \par

КНА работи по следния начин: Когато в резултат на изчисление на фунцкията $\delta$ се получи множеството $Q'$ от състояния в които КНА трябва да премине, той се размножава в $|Q'|$ копия и всяко копие преминава в едно от състоянията на $Q'$. \textbf{Можем да си мислим, че автоматът едновременно се намира във всичките състония на $Q'$.} Когато КНА прочете следващата буква, всяко от копията се разможава в толкова нови копия, колкото текущото му състояние, входната буква и функцията $\delta$ определят и всяко едно копие преминава в съответното състояние. Множеството от състояния в които се намира КНА ще съдържа състоянията на всички получени копия. Тази дефиниция ще уточним формално по следния начин. \par

\italicBold{Дефиниция:} Нека $\kda$ е КНА. Разширена функция на преходите $\Delta : Q \times X^{*} \to 2^{Q}$ на $A$ дефинираме така:\\
$\Delta(q, \epsilon) = \qquad \qquad \qquad \qquad \{q\}\\
\Delta(q, x) =
	\begin{cases}
		\delta(q,x) \qquad \qquad \; \forall q,x, \text{, за които } 				\delta \text{ е дефинирана} \\ 
		\text{недефинирана} \quad \text{ако } \delta(q,x) \text{не е 				дефинирана} 
	\end{cases}\\
\Delta(q, \alpha x) = U_{i = 1}^{l} \delta(q^{pi}, x) \quad \quad \Delta(q, \alpha) = \{q_{p1}, q_{p2},..., q_{pn}\}.$  \par

\italicBold{Дефиниция:} Казваме, че КНА $A$ разпознава думата $\alpha \belongsTo X^{*}$, ако $\Delta(q_{0}, \alpha) \cap F \neq \emptyset$. Език, разпознаван от КНА $А$ определяме като \\
\centerline{$L_{A} = \{\alpha|\alpha \belongsTo X^{*}, \Delta(q_{0}, \alpha)\cap F \neq \emptyset \}$}

\theorem \emph{За всяка автоматна граматика $\Gamma$ съществува КНА $A$, такъв че $L_{\Gamma} = L_{A}$.}\\
\centerline{\emph{Insert proof here:}}

\theorem \emph{За всеки КНА $A$ съществува КДА $A'$ такъв, че $L_{A} = L_{A'}$}

\proof Нека $\kda$. Построяваме КДА 
$A' = < Q', X, t_{0}, \delta', F'$, където $Q' \subseteq 2^{Q}$. Нека множеството 
$\{q_{p1}, q_{p2}, ..., q_{pl} \} \belongsTo Q'$. За по кратко ще го означаваме с 
$t_{[p1, p2,..., pl ]}$. При това изначение определяме 
$t_{0} = \{q_{0}\} = t_{|0|}.$. Нека 
$F' = \{t_{[p1, p2, ..., pl]}|\{ q_{p1}, q_{p2}, ..., q_{pl} \}
\cap F \neq \emptyset\}, a \delta'(t_{[p1, p2, ... ,pl]}, x) = 
t_{[r1, r2, ..., rm]}, \text{ ако } \{q_{r1}, q_{r2}, ..., q_{rm}\}
= \cup_{i = 1}^{l}\delta(q_{pi}, x)$. Забележете, че не можем да фиксираме в явен вид кои точно подмножества на $Q$ влизат в $Q'$. Те се определят от изчисляването на функцията $\delta'$, започвайки от $\delta'(t_{|0|}, \alpha), \forall x \in X$. 
С индукция по дължината на $\alpha$ ще покажем, че $\Delta_{A'}(t_{|0|}, \alpha) = t_{[p1, p2, ..., pl]} \Leftrightarrow 
\Delta_{A}(q_{0}, \alpha) = \{q_{p1}, q_{p2}, ..., q_{pl}\}$.

\renewcommand{\theenumi}{\alph{enumi}}
\begin{enumerate}
	\item Нека $\omega = \epsilon$. Тогава $\Delta_{A'}(t_{|0|}, 				\epsilon) = t_{|0|}, a \Delta_{A}(q_{0}, \epsilon)= \{q_{0}				\}$ и твърдението е в сила
	\item Допускаме верността на твърдението за някоя дума $\alpha$ 			и ще покажем, че $\Delta_{A'}(t_{t|0|}, \alpha x) = t_{[r1, 			r2, ..., rm]} \leftrightarrow \Delta_{A}(q_{0}, \alpha x) = 			\{q_{r1}, q_{r2}, ..., q_{rm}\}$.
	\item Нека $\Delta_{A'}(t_{|0|}, \alpha x) = t_{[r1, r2, ..., 				rm]}$, като
		$\Delta_{A'}(t_{|0|}, \alpha) = t_{[p1, p2, ..., pl]}$. 				Тогава $\delta'(t_{[p1, p2, ..., pl]}, x) = t_{[r1, r2, ..., 		rm]}$ и съгласно построението на $\delta'\{q_{r1}, q_{r2}, 				..., q_{rm} \} = $
		$\cup_{i=1}^{l}\delta(q_{pi}, x).$ Но от индуктивното 					допускане 
		$\Delta_{A}(q_{0}, \alpha) = \{q_{p1}, q_{p2}, ..., q_{pl}\}			$ и следователно 
		$\Delta_{A}(q_{0}, \alpha x) = \{q_{r1}, q_{r2}, ..., q_{rm}			\}$. Разсъжденията в другата посока са подобни.
\end{enumerate} \par

Като вземем предвид дефиницията на $F'$ и току що доказаното, получаваме $\Delta_{A'}(t_{|0|}, \alpha) \in F' \leftrightarrow 
\Delta_{A}(q_{0}, \alpha) \cap F \neq \emptyset$ и значи 
$L_{A'} = L_{A}$   
		
  


\end{document}
