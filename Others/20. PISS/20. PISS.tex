% !TEX TS-program = pdflatex
% !TEX encoding = UTF-8 Unicode

% This is a simple template for a LaTeX document using the "article" class.
% See "book", "report", "letter" for other types of document.

\documentclass[11pt]{article} % use larger type; default would be 10pt

\usepackage[utf8]{inputenc} % set input encoding (not needed with XeLaTeX)

%%% Examples of Article customizations
% These packages are optional, depending whether you want the features they provide.
% See the LaTeX Companion or other references for full information.

%%% PAGE DIMENSIONS
\usepackage{geometry} % to change the page dimensions
\geometry{a4paper} % or letterpaper (US) or a5paper or....
% \geometry{margin=2in} % for example, change the margins to 2 inches all round
% \geometry{landscape} % set up the page for landscape
%   read geometry.pdf for detailed page layout information

\usepackage{graphicx} % support the \includegraphics command and options

% \usepackage[parfill]{parskip} % Activate to begin paragraphs with an empty line rather than an indent

%%% PACKAGES
\usepackage{booktabs} % for much better looking tables
\usepackage{array} % for better arrays (eg matrices) in maths
\usepackage{paralist} % very flexible & customisable lists (eg. enumerate/itemize, etc.)
\usepackage{verbatim} % adds environment for commenting out blocks of text & for better verbatim
\usepackage{subfig} % make it possible to include more than one captioned figure/table in a single float
% These packages are all incorporated in the memoir class to one degree or another...

%%% HEADERS & FOOTERS
\usepackage{fancyhdr} % This should be set AFTER setting up the page geometry
\pagestyle{fancy} % options: empty , plain , fancy
\renewcommand{\headrulewidth}{0pt} % customise the layout...
\lhead{}\chead{}\rhead{}
\lfoot{}\cfoot{\thepage}\rfoot{}

%%% SECTION TITLE APPEARANCE
\usepackage{sectsty}


\allsectionsfont{\sffamily\mdseries\upshape} % (See the fntguide.pdf for font help)
% (This matches ConTeXt defaults)

%%% ToC (table of contents) APPEARANCE
\usepackage[nottoc,notlof,notlot]{tocbibind} % Put the bibliography in the ToC
\usepackage[titles,subfigure]{tocloft} % Alter the style of the Table of Contents
\renewcommand{\cftsecfont}{\rmfamily\mdseries\upshape}
\renewcommand{\cftsecpagefont}{\rmfamily\mdseries\upshape} % No bold!

%%% END Article customizations


\usepackage[bulgarian]{babel}
\usepackage{physics}
\usepackage{amsmath}
\usepackage{centernot}
\usepackage{url}
\usepackage{graphicx}
\graphicspath{ {.} }
\usepackage{amsfonts}
\usepackage{xcolor}
\usepackage{enumitem}
\usepackage{systeme}
\usepackage{listings}
\usepackage[cache=false]{minted}
\usepackage{csquotes}
\setquotestyle{english}



%%% The "real" document content comes below...

\title{20. Проектиране и интегриране на софтуерни системи}
\author{Play4u}
%\date{} % Activate to display a given date or no date (if empty),
         % otherwise the current date is printed
         

\newcommand{\lrangle}[1]{\left\langle #1 \right\rangle}

\newcommand{\oversetModels}[1]{\overset{#1}{\models}}

\newcommand{\italicBold}[1]{\textbf{\emph{#1}}}

\newcommand{\definition}{\italicBold{Дефиниция: }}
\newcommand{\theorem}{\italicBold{Теорема: }}
\newcommand{\lemma}{\italicBold{Лема: }}
\newcommand{\proof}{\italicBold{Доказателство: }}
\newcommand{\statement}{\italicBold{Твърдение: }}
\newcommand{\source}{\italicBold{Източник: }}

\newcommand{\integral}[4]{\displaystyle \int_{#1}^{#2}#3\,#4}

\newcommand{\redText}[1]{\textcolor{red}{#1}}

\newcommand{\curlies}[1]{\{#1\}}
\newcommand{\overbar}[1]{\mkern 1.5mu\overline{\mkern-1.5mu#1\mkern-1.5mu}\mkern 1.5mu}


\newcommand{\enumNum}{\renewcommand{\theenumi}{\arabic{enumi}}}
\newcommand{\enumlet}{\renewcommand{\theenumi}{\alph{enumi}}} 

\begin{document}
\maketitle

\italicBold{Конспект: } Изложението на въпроса трябва да включва следните по-съществени елементи:

\enumNum
\begin{enumerate}[noitemsep]
	\item Характеристика на разпределените софтуерни системи - дефиниции, видове системи и тенденции
	\item Междупроцесна комуникация - отдалечено извикване, млултикаст
	\item Разпределени обекти и компоненти
	\item Уеб услуги - дефиниции, шаблони за комникация. Стандарти за уеб услуги - SOAP, UDDI, WSDL.\\\par
\end{enumerate}


\section{Характерстики}
\subsection{Дефниции}
Разпределена система е такава, чиито компоненти са разположени на различни, свързани чрез мрежа, компютри, които си комуникират и се координират чрез размяна на съобщения помежду си. - Wikipedia\\
Разпределената система представлява съвкупност от компютри, които изглеждат на потребителите си като една кохерентна система. - Таненбаум\\
Разпределената система е съвкупност от автономни хостове, които са свързани чрез компютърна мрежа. Всеки хост изпълнява компоненти и върху него работи разпределен middleware, който позволява компонентите да координират своите активности по такъв начин, че потребителите да възприемат системата като единично, интегрирано средство за работа. - Емерих\\\par

\italicBold{Middleware}\\
С усложняването на софтуера се налага и по-голяма абстракция и все по-сложни примитиви за разработка. Така се преминава от операционни системи към мрежови операционни системи докато се стига до нуждата от нов абстрактен слой, който да предоставя основни услуги и който има за цел прозрачност на разпределеността на системата. Този слой се нарича middleware.\\
\definition Middleware е термин, който се отнася до множеството от софтуерни услуги, което се намира между приложението и операционната система и има за цел да улесни разработването на разпределени приложения чрез абстрахиране от сложеността и хетерогенността на намиращата се отдолу сред операционни системи, хардуерни платформи и комуникационни протоколи.\\\par

\subsection{Разпределени системи}\\
\begin{itemize}[noitemsep]
	\item \textbf{Client-Server: } В този тип системи клиентското приложение изисква някакъв ресурс, който сървъра предоставя. Един сървър може да обслужва множество клиенти едновременно, за разлика от клиента, който може да е в контакт само с един сървър. Обикновено и клиентът, и сървърът са в една и съща компютърна мрежа, затова се счита, че те са част от разпределените системи. 
	\item \textbf{Peer-to-Peer: } Тези системи са изградени от върхове(nodes) - компютри - които имат еднакъв статут при споделянето на данни помежду си. Всички задачи се разпределят равномерно сред върховете. Те взаимодействат един с друг, споделяйки ресурси. Това се случва с помощта на мрежата с която са свързани.  
	\item \textbf{Трислойни: } Информацията на клиента се съдържа в средния слой, вместо в клиентския слой, с цел улесняване на интеграцията на приложението. Този архитектурен модел се среща най-честно сред уеб приложенията.
	\item \textbf{N-слойни: } В най-общия случай се изплолзва , когато приложение или сървър трябва да препрати заявки към други бизнес услуги на същата мрежа. \\\par
\end{itemize}

\subsection{Тенденции}
\source \path{https://www.quora.com/What-are-the-trends-in-distributed-systems}
\begin{itemize}[noitemsep]
	\item The emergence of pervasive networking technology
	\item The emergence of ubiquitous computing
	\item The increase in demand for multimedia services
	\item View of DS as the utility
\end{itemize}

  
\end{document}






