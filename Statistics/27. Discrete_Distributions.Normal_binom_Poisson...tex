% !TEX TS-program = pdflatex
% !TEX encoding = UTF-8 Unicode

% This is a simple template for a LaTeX document using the "article" class.
% See "book", "report", "letter" for other types of document.

\documentclass[11pt]{article} % use larger type; default would be 10pt

\usepackage[utf8]{inputenc} % set input encoding (not needed with XeLaTeX)

%%% Examples of Article customizations
% These packages are optional, depending whether you want the features they provide.
% See the LaTeX Companion or other references for full information.

%%% PAGE DIMENSIONS
\usepackage{geometry} % to change the page dimensions
\geometry{a4paper} % or letterpaper (US) or a5paper or....
% \geometry{margin=2in} % for example, change the margins to 2 inches all round
% \geometry{landscape} % set up the page for landscape
%   read geometry.pdf for detailed page layout information

\usepackage{graphicx} % support the \includegraphics command and options

% \usepackage[parfill]{parskip} % Activate to begin paragraphs with an empty line rather than an indent

%%% PACKAGES
\usepackage{booktabs} % for much better looking tables
\usepackage{array} % for better arrays (eg matrices) in maths
\usepackage{paralist} % very flexible & customisable lists (eg. enumerate/itemize, etc.)
\usepackage{verbatim} % adds environment for commenting out blocks of text & for better verbatim
\usepackage{subfig} % make it possible to include more than one captioned figure/table in a single float
% These packages are all incorporated in the memoir class to one degree or another...

%%% HEADERS & FOOTERS
\usepackage{fancyhdr} % This should be set AFTER setting up the page geometry
\pagestyle{fancy} % options: empty , plain , fancy
\renewcommand{\headrulewidth}{0pt} % customise the layout...
\lhead{}\chead{}\rhead{}
\lfoot{}\cfoot{\thepage}\rfoot{}

%%% SECTION TITLE APPEARANCE
\usepackage{sectsty}


\allsectionsfont{\sffamily\mdseries\upshape} % (See the fntguide.pdf for font help)
% (This matches ConTeXt defaults)

%%% ToC (table of contents) APPEARANCE
\usepackage[nottoc,notlof,notlot]{tocbibind} % Put the bibliography in the ToC
\usepackage[titles,subfigure]{tocloft} % Alter the style of the Table of Contents
\renewcommand{\cftsecfont}{\rmfamily\mdseries\upshape}
\renewcommand{\cftsecpagefont}{\rmfamily\mdseries\upshape} % No bold!

%%% END Article customizations


\usepackage[bulgarian]{babel}
\usepackage{physics}
\usepackage{amsmath}
\usepackage{centernot}
\usepackage{url}
\usepackage{graphicx}
\graphicspath{ {.} }
\usepackage{amsfonts}
\usepackage{xcolor}
\usepackage{enumitem}
\usepackage{csquotes}
\setquotestyle{english}
\usepackage{systeme}


%%% The "real" document content comes below...

\title{27. Дискретни разпределения. Равномерно, биномно, геометрично, Поансово}
\author{Play4u}
%\date{} % Activate to display a given date or no date (if empty),
         % otherwise the current date is printed
         

\newcommand{\lrangle}[1]{\left\langle #1 \right\rangle}

\newcommand{\oversetModels}[1]{\overset{#1}{\models}}

\newcommand{\italicBold}[1]{\textbf{\emph{#1}}}

\newcommand{\definition}{\italicBold{Дефиниция: }}
\newcommand{\theorem}{\italicBold{Теорема: }}
\newcommand{\lemma}{\italicBold{Лема: }}
\newcommand{\proof}{\italicBold{Доказателство: }}
\newcommand{\statement}{\italicBold{Твърдение: }}
\newcommand{\source}{\italicBold{Източник: }}

\newcommand{\integral}[4]{\displaystyle \int_{#1}^{#2}#3\,#4}

\newcommand{\redText}[1]{\textcolor{red}{#1}}

\newcommand{\curlies}[1]{\{#1\}}
\newcommand{\overbar}[1]{\mkern 1.5mu\overline{\mkern-1.5mu#1\mkern-1.5mu}\mkern 1.5mu}

\newcommand{\enumNum}{\renewcommand{\theenumi}{\arabic{enumi}}}
\newcommand{\enumlet}{\renewcommand{\theenumi}{\alph{enumi}}} 

\begin{document}
\maketitle

\italicBold{Конспект: } На изпита комисията дава две разпределения, върху които се развива въпросът. Дефиницията на дискретно вероятностно разпределение на случайна величина. Свойства на вероятностите (неотрицателност и норминарост, монотонност и адитивност). За всяко от дадените две разпределения да се посочи пример, при който то възниква. Да се пресметне математическото очакване и дисперсията на всяко от тези очаквания. При пресмятане може да се използва пораждаща функция или пораждаща моментите функция, но тя трябва да се дефинира и да се покажат основните й свойства (без доказателство).


\section{Дискретни разпределения}
\source \path{http://qub.ac.uk/helm/HELM_2008/pages/workbooks_1_50_jan2008/Workbook37/37_1_dscrt_prob_distn.pdf}\\\par

Тройката $(\Omega, \aleph, P)$, където $(\Omega, \aleph(\Omega))$ е измеримо пространство и $P$ е вероятностна мярка, дефинирана върху него юе наричаме \textit{вероятностно пространство}\\\par

Случайна величина наричаме всяка функция $\xi : \Omega \to \mathbb{R}$, такава, че за произволно реално $x(\forall x \in \mathbb{R})$, множеството $\curlies{\omega:\xi(\omega)\leq x}\in \mathfrak{F}$, т.е. множествено събитие.\\\par

\definition Разпределение на дискретна случайна величина:\\
Случайна величина $X$ и нейните разпределения се наричат \textit{дискретни} ако стойностите на $X$ могат да бъдат представени като наредена редица, пр. $x_{1}, x_{2}, x_{3}...$, със стойности на вероятността $p_{1}, p_{2}, p_{3}...$. Тоест $P(X=x_{i})=p_{i}$. По общо, дискретното разпределение $f(x)$ може да бъде дефинирано чрез:\\
\centerline{$f(x)=
\begin{cases} 
      p_{i} & \text{ ако } x=x_{i} \qquad i = 1,2,3... \\
      0 & \text{иначе}
\end{cases}$}

Функцията на разпределението $F(x)$ се получава чрез сумите дефинирани по следния начин:\\
\centerline{$F(x)=\displaystyle \sum_{x_{i} \leq x}f(x_{i})=\sum_{x_{i}\leq x}p_{i}$}
Сумираме вероятностите $p_{i}$ за които $x_{i}$ е по-малко или равна на $x$. Това ни дава прекъсната функция със "стъпки" с размер $p_{i}$, за всяка стойност $x_{i}$ от $X$. Тази функция е дефинирана за всички стойности, не само за стойностите $x_{i}$ от $X$\\

Малко по-сбито(Научи това):\\
Нека $X$ е случайна величина, съответваща на експеримент. Нека стойностите на $X$ са обозначени от $x_{1}, x_{2},...,x_{n}$ и нека $P(X=x_{i})$ е вероятността $x_{i}$ да се случи. Имаме две необходими условия за валидно дискретно разпределение:\\
\begin{itemize}
	\item $P(X = x_{i}) \geq 0 \; \forall \; x_{i}$\\
	\item $\displaystyle \sum_{i=1}^{n}P(X=x_{i})=1$\\
\end{itemize}
Трябва да отбележим, че $n$ може да е безкрайно.\\
(Тези две твърдения са достатъчни да гарантираме, че $P(X=x_{i})\leq 1 \; \forall \; x_{i}$)

\section{Свойства на вероятностите}
\source \path{http://fmi.wikidot.com/prob1#toc17}

\subsection{Неотрицателност:} 
Всяка вероятност е от 0 до 1.\\
0 означава невъзможно, 1 означава 100\% сигурно, т.е.\\
\centerline{$\forall \; E \to 0 \leq P(E) \leq 1$}

\subsection{Норминарност:}
Вероятността на достоверно събитие е едно. т.е.\\
\centerline{$P(\Omega) = 1$}

\subsection{Адитивност:}
Имаме изброимо безкрайна редица от несъвместими събития $E_{1}, E_{2},...,E_{n}$. Вероятността на тяхното обединение се дава по следния начин:\\
\centerline{$P\displaystyle(\sum_{i=1}^{\infty}E_{i})=\sum_{i=1}^{\infty}P(E_{i})$}

\subsection{Монотонност}
Ако $A,B \in \aleph(\Omega)$ и $A \subseteq B$, то\\
\centerline{$P(A) \leq P(B)$}

\section{Конкретни дискретни разпределния}
\source \path{http://pharmfac.net/social_pharm_lectures/TWMS_L/L07.pdf}\\

\subsection{Пораждаща функция}
\source \path{https://www.cl.cam.ac.uk/teaching/0708/Probabilty/prob06.pdf}\\
Генериращата вероятности функция на дискретна произволна величина представя масовата функция(probablity mass function) на величината чрез полином, чиито коефициенти са вероятностите да се случи определения изход, а всяка една от променливите, асоциирани със съответните коефициенти са от степен, с 1 по-голяма от предишната, започвайки от 0. Тоест, ако $X$ е произволна величина, тогава пораждащата функция на $X$ се определя като:\\
\centerline{$G(\eta)=\displaystyle \sum_{x=0}^{\infty} p(x)z^{x}$} 
Две от по-важните свойства на пораждащите функции са:
\begin{itemize}[noitemsep]
	\item $E(X)=G'(1)$
	\item $Var(X)=G''(1)+G'(1)-[G(1)^{2}]$
\end{itemize} 

\subsection{Равномерно разпределение}
Нека $a<b$
\definition Непрекъснатата случ. величина(с.в.) $X$ се нарича \textit{равномерно разпределена в интервала} $[a,b]$, ако има плътност на разпределение \\
\centerline{$p(x)=
\begin{cases} 
      \frac{1}{b-a} & \text{ ако } x \in [a,b],\\
      0 & \text{ ако } x \not\in [a,b].
\end{cases}$}
Фунцкията на разпределение на такава с.в. е \\
$F(x) = 
\begin{cases} 
      0 & \text{ ако } x < a,\\
      \frac{x-a}{b-a} & \text{ ако } a \leq x \leq b,\\
      1 & \text{ ако } x > b.
\end{cases}$\\
\textbf{$E(X):$} $E(X)=\integral{a}{b}{\dfrac{x}{b-a}{dx}}=\dfrac{1}{b-a}\integral{a}{b}{x}{dx}=\dfrac{1}{b-a}\dfrac{x^{2}}{2}\bigg\rvert^{b}_{a}=\dfrac{1}{b-a}\dfrac{b^{2}}{2}-\dfrac{a^{2}}{2}=$\\
\centerline{$=\dfrac{(b-a)(b+a)}{2(b-a)}=\dfrac{b+a}{2}$}
$var(X)=E(x^{2})-[E(X)]^{2}$\\
$E(X^{2})=\dfrac{1}{b-a}\integral{a}{b}{x^{2}}{dx}=\dfrac{1}{b-a}\dfrac{x^{3}}{3}\bigg\rvert^{b}_{a}=\dfrac{b^{3}-a^{3}}{3(b-a)}=\dfrac{(a^{2}+ab+b^{2})(b-a)}{3(b-a)}=\dfrac{a^{2}+2ab+b^{2}}{3}$\\
$var(X)=\dfrac{a^{2}+ab+b^{2}}{3}-\dfrac{a^{2}+2ab+b^{2}}{4}=\dfrac{a^{2}-2ab+b^{2}}{12}=\dfrac{(a-b)^{2}}{12}$\\
$\sigma(X)=\sqrt{var(X)}=\dfrac{|a-b|}{\sqrt{12}}$
\newpage
Равномерно разпределените случайни величини намират приложение: 
\begin{itemize}
	\item в задачи от геометрична вероятност\\
	\item за формиране на непрекъсната с.в. $Y$ с помощта на с.в. $X$, която е равномерно разпределена в интервала $(0,1)$ и се получава на компютър със стандартна програма "генератор на случайни числа"\\
\end{itemize}
Пр.: хвърляне на честен зар. Тогава
\begin{itemize}
	\item $X=$ шанса да се падне ези\\
	\item $X=$ шанса да се падне четно число\\
	\item $X=$ шанса да се падне число, по малко от 3\\
\end{itemize}

\subsection{Биномно разпределение}
Нека $n \geq 1$ е цяло число, $p \in [0,1]$ и $q=1-p$.\\
\definition Дискретната с.в. $X$ има \textit{биномно разпределние} с параметри $n, p$, ако приема стойности $0,1,...,m,...,n$ с вероятности\\
\centerline{$p_{m}=P(X=m)=P_{n}(n)=C_{n}^{m}p^{m}q^{n-m}$.}\\\\
За пораждащата ф-ция на разпределението имаме:\\
$G(\eta)=(q+p\eta)^{n}$\\
$G'(\eta)=n(q+p\eta)^{n-1}p$
$G''(\eta)=n(n-1)(q+p\eta)^{n-2}p^{2}$.\\ 
Така:\\
$E(X)=G'(1)=n(q+p)^{n-1}p$ и тъй като $q=1-p \rightarrow q+p=1$ получаваме $E(X)=G'(1)=n1^{n-1}p=np$\\
$var(X)=G''(1)+G'(1)-[G'(1)^{2}]=n(n-1)p^{2}+np-n^{2}p^{2}=np(1-p)=npq$\\
$\sigma(X)=\sqrt{npq}$\\

Тълкуване: Провеждат се $n$ опита по схемата на Бернули с вероятност $p$ за успех при един опит. Тогава броят на успешните опити $X$ е случайна величина, която има биномно разпределение с параметри $n, p$.\\
Пр.:
\begin{itemize}
	\item Монета е хвърлена 10 пъти. Каква е вероятността да се падне ези точно 6 пъти?\\
	\item 80\% от хората, които има застраховка на домашните си любимци са жени. Ако на произволен принцип са избрани 9 човека, които са си купили застраховка за домашния си любимец, какъв е шанса точно 6 от тях да са жени? 
	80% of people who purchase pet insurance are women.  If 9 pet insurance owners are randomly selected, find the probability that exactly 6 are women
\end{itemize}


 
\subsection{Геометрично разпределение}
Нека $p \in (0,1)$\\
\definition Дискретната с.в. $X$ има \textit{геометрично разпределние} с параметър $p$, ако приема стойности $1,2,...,n,...$ с вероятности\\
\centerline{$p_{n}=P(X=n)=(1-p)^{n-1}p$.}\\
$G(\eta)=p(1-q\eta)^{-1}, \quad |s|<\frac{1}{q}$\\
$G'(\eta)=-p(1-q\eta)^{-2}-q=pq(1-q\eta)^{-2}$\\
$G''(\eta)=-2pq(1-q\eta)^{-3}-q=2pq^{2}(1-q\eta)^{-3}$\\
Така ако геометричното разпределение е дефинирано като \enquote{брой неуспешни Бернулеви опити до първия успешен опит} имаме:\\
$E(X)=G'(1)=pq(1-q)^{-2}=\dfrac{pq}{p^{2}}$(тъй като $p=1-q$)$=\dfrac{q}{p}=\dfrac{1-p}{p}$\\
$var(X)=G''(1)+G'(1)-[G'(1)^{2}]=2pq^{2}(1-q)^{-3}+\dfrac{1+p}{p}-\dfrac{q^{2}}{p^{2}}=\dfrac{q^{2}+q-q^{2}}{p^{2}}=\dfrac{q}{p^{2}}=\dfrac{1-p}{p^{2}}$\\
$\sigma(X)=\dfrac{\sqrt{1-p}}{p}$\\

Ако пък търсим общия брой опити до настъпването на първия успех(т.е. всички провали + първия успех) случайната величина става\\
\centerline{$Y=X+1$}
Тогава от св-вото, че $E(X+\alpha)=E(X)+\alpha$, където $\alpha=const$, имаме че:\\
$E(Y)=E(X+1)=E(X)+1=\dfrac{1-p}{p}+1=\dfrac{1}{p}$\\
А от св-вото, че $var(X+\alpha)=var(X)$, имаме че:\\
$var(Y)=var(X+1)=var(X)=\dfrac{1-p}{p^{2}}$\\
$\sigma(X)=\dfrac{\sqrt{1-p}}{p}$\\

\textit{Тълкуване:} Провежда се редица от независими опити с вероятност за успех при един опит $p$. Тогава номерът $X$ на първия успешен опит е случайна величина, която има геометрично разпределение.\\
Примери:
\begin{itemize}
	\item Хвърляме честна монета няколко пъти. Тогава \\
		\centerline{$X=$ броя хвърляния до първото падане на ези}
	\item Знаем, че 20\% от прдуктите върху една продуктова линия са дефектни. Изследваме продуктите, докато не намерим първия дефектен продукт. Тогава \\
		\centerline{$X=$ броя изследвани продукти до намиране на първия дефектен}
\end{itemize}

\subsection{Поасоново разпределение}
Нека $\lambda > 0$.\\
\definition Дискретната с.в. $X$ има \textit{разпределение на Поасон} с параметър $\lambda$, ако приема стойности $0,1,...,m,...$ с вероятности\\
\centerline{$p_{m}=P(X=m)=\frac{\lambda^{m}}{m!}e^{-\lambda}$.}\\\\
$G(\eta)=e^{(\lambda\eta)}e^{-\lambda}$\\
$G'(\eta)=\lambda e^{(\lambda\eta)}e^{-\lambda}$\\
$G''(\eta)=\lambda^{2}e^{(\lambda\eta)}e^{-\lambda}$\\
Така:\\
$E(X)=G'(1)=\lambda e^{\lambda} e^{-\lambda}=\lambda$\\
$var(X)=G''(1)+G'(1)-[G'(1)^{2}]=\lambda^{2}+\lambda -\lambda^{2}=\lambda$\\
$\sigma(X)=\sqrt{\lambda}$\\

Пр.: Случайни величини, които имат разпределение на Поасон, се срещат често в теорията на масовото обслужване, теорията на надеждността, теорията на управление на запасите, биологията, физиката и др.
рията на управление на запасите, биологията, физиката и др.
Типични примери за случайни величини, имащи разпределение на
Поасон са следните:
\begin{itemize}
	\item брой на космически частици, които попадат през определен период от време върху повърхнината на датчик за регистриране на такива\\
	\item брой на телефонните повиквания, постъпващи за определено време\\
	\item брой на нишките, които са се скъсали на тъкачен стан за определено време
\end{itemize}


\end{document}




